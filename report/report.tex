\documentclass[a4paper]{article}
\usepackage[margin=1.25in]{geometry}
\usepackage{bookmark}

\usepackage{amsmath}
\usepackage{amssymb}
\allowdisplaybreaks
\newcommand{\numberthis}{\addtocounter{equation}{1}\tag{\theequation}}
\newcommand{\labeleqn}[1]{\numberthis \label{#1}}
\newcommand{\case}[1]{\numberthis \vspace*{10px} \textit{Case #1} \hspace*{2.5px}}

\usepackage{graphicx}
\usepackage{float}

\renewcommand{\baselinestretch}{1.15}
\setlength{\parindent}{0pt}

\usepackage{multirow}
\usepackage{tabularx}
\newcolumntype{L}{>{\centering\arraybackslash}X}

\usepackage[skins]{tcolorbox}
\newtcolorbox{plainbox}[2][]{%
  enhanced,colback=white,colframe=black,coltitle=black,
  sharp corners,boxrule=0.4pt,
  fonttitle=\itshape,
  attach boxed title to top center={yshift=-0.4\baselineskip-0.4pt},
  boxed title style={tile,size=minimal,left=0.5mm,right=0.5mm,
    colback=white,before upper=\strut},
  title=#2,#1
}

\title{CS1.404: Assignment 3}
\author{Himanshu Singh}
\date{\today}

\begin{document}

\maketitle

\begin{plainbox}{KKT Condition}
For the convex optimization problem,
\begin{align*}
\min_{\textbf{x} \in \mathbb{R}^n} f(\textbf{x}) \\
\text{s.t. } h_j(\textbf{x}) \leq 0 && \forall j \in \{1, 2, \ldots, l\}
\intertext{$\textbf{x}^*$ is an optimal solution, if there exists multipliers $\{\lambda_j\}_{j = 1}^l$, such that the following equations hold.}
\nabla f(\textbf{x}^*) + \sum_{j = 1}^l \lambda_j \nabla h_j (\textbf{x}^*) = \textbf{0} \\
\lambda_j h_j (\textbf{x}^*) = 0 && \forall j \in \{1, 2, \ldots, l\} \\
\lambda_j \geq 0 && \forall j \in \{1, 2, \ldots, l\} \labeleqn{kkt}
\end{align*}
\end{plainbox}

\section{Trid Function}

$$f(\textbf{x}) = \sum_{i=1}^d (x_i - 1)^2 - \sum_{i=2}^d x_{i-1} x_i$$

\begin{align*}
\nabla f(\textbf{x}) &=
    \begin{bmatrix}
        2 x_1 - x_2 - 2 \\
        2 x_2 - x_1 - x_3 - 2 \\
        2 x_3 - x_2 - x_4 - 2 \\
        \vdots \\
        2 x_{d-1} - x_{d-2} - x_d - 2 \\
        2 x_d - x_{d-1} - 2 \\
    \end{bmatrix}
\end{align*}

\subsection{Test Case 0}

\begin{align*}
h_1(\textbf{x}) &= x_1^2 - 2 x_2 \\
\nabla h_1(\textbf{x}) &= \begin{bmatrix} 2 x_1 & -2 \end{bmatrix} ^T \\
\end{align*}

Substituting in \eqref{kkt}, we get
\begin{align*}
2 x_1 - x_2 - 2 + 2 \lambda_1 x_1 &= 0 \\
2 x_2 - x_1 - 2 - 2 \lambda_1 &= 0 \\
\lambda_1 (x_1^2 - 2 x_2) &= 0 \labeleqn{tc0}
\end{align*}

\case{1} When $\lambda_1 = 0$, the system \eqref{tc0} reduces to
\begin{align*}
2 x_1 - x_2 - 2 &= 0 \\
2 x_2 - x_1 - 2 &= 0 \labeleqn{tc0-case1}
\end{align*}
Solving \eqref{tc0-case1}, we get $x_1 = x_2 = 2$.

\case{2} When $\lambda_1 \neq 0$, the system \eqref{tc0} reduces to
\begin{align*}
2 x_1 - x_2 - 2 + 2 \lambda_1 x_1 &= 0 \\
2 x_2 - x_1 - 2 - 2 \lambda_1 &= 0 \\
x_1^2 - 2 x_2 &= 0 \labeleqn{tc0-case2}
\end{align*}

Eliminating $x_2$ from \eqref{tc0-case2}, we get
\begin{align*}
2 x_1 - \frac{x_1^2}{2} - 2 + 2 \lambda_1 x_1 &= 0 \\
x_1^2 - x_1 - 2 - 2 \lambda_1 &= 0 \labeleqn{tc0-case2-2}
\end{align*}

Eliminating $\lambda_1$ from \eqref{tc0-case2-2}, we get
\begin{align*}
x_1^3 - \frac{3x_1^2}{2} - 2 = 0 \labeleqn{tc0-case2-3}
\end{align*}

Solving \eqref{tc0-case2-3} we get $x_1 = 2$ as the only real solution. Substituting back in \eqref{tc0-case2-2}, we get $\lambda_1 = 0$, contradicting our assumption that $\lambda_1 \neq 0$.

We thus conclude that $(\textbf{x}^*, \lambda^*) = \left( \begin{bmatrix} 2 & 2 \end{bmatrix} ^T, 0 \right)$ is the only real solution to \eqref{tc0}.

\subsection{Test Case 1}

\begin{align*}
h_1(\textbf{x}) &= x_1^2 - x_2^2 + 1 \\
\nabla h_1(\textbf{x}) &= \begin{bmatrix} 2 x_1 & -2 x_2 \end{bmatrix} ^T \\
\end{align*}

Substituting in \eqref{kkt}, we get
\begin{align*}
2 x_1 - x_2 - 2 + 2 \lambda_1 x_1 &= 0 \\
2 x_2 - x_1 - 2 - 2 \lambda_1 x_2 &= 0 \\
\lambda_1 (x_1^2 - x_2^2 + 1) &= 0 \labeleqn{tc1}
\end{align*}

\case{1} When $\lambda_1 = 0$, the system \eqref{tc1} reduces to
\begin{align*}
2 x_1 - x_2 - 2 &= 0 \\
2 x_2 - x_1 - 2 &= 0 \labeleqn{tc1-case1}
\end{align*}
Solving \eqref{tc1-case1}, we get $x_1 = x_2 = 2$.

\case{2} When $\lambda_1 \neq 0$, the system \eqref{tc1} reduces to
\begin{align*}
2 x_1 - x_2 - 2 + 2 \lambda_1 x_1 &= 0 \\
2 x_2 - x_1 - 2 - 2 \lambda_1 x_2 &= 0 \\
x_1^2 - x_2^2 + 1 &= 0 \labeleqn{tc1-case2}
\end{align*}

Eliminating $x_2$ from \eqref{tc1-case2}, we get
\begin{align*}
2 x_1 \mp \sqrt{x_1^2 + 1} - 2 + 2 \lambda_1 x_1 &= 0 \\
\pm 2 \sqrt{x_1^2 + 1} - x_1 - 2 \mp 2 \lambda_1 \sqrt{x_1^2 + 1} &= 0 \labeleqn{tc1-case2-2}
\end{align*}

Eliminating $\lambda_1$ from \eqref{tc1-case2-2}, we get
\begin{align*}
\mp \frac{x_1^2 + 2x_1}{\sqrt{x_1^2 + 1}} + 4x_1 \mp \sqrt{x_1^2 + 1} - 2 &= 0 \labeleqn{tc1-case2-3}
\end{align*}

Solving \eqref{tc1-case2-3} we get $x_1 = \{1.8991, 0.16094\}$ as the real solutions. Substituting back in \eqref{tc1-case2-2}, we get $\lambda_1 = \{0.90167, 2.06674\}$ respectively. Substituting back in \eqref{tc1-case2}, we get $x_2 = \{2.14629, -1.01287\}$ respectively.

We thus conclude that $(\textbf{x}^*, \lambda^*) = \left\{ \left( \begin{bmatrix} 2 & 2 \end{bmatrix} ^T, 0 \right), \left( \begin{bmatrix} 1.8991 & 2.14629 \end{bmatrix} ^T, 0.09167 \right), \left( \begin{bmatrix} 0.16094 & -1.01287 \end{bmatrix} ^T, 2.06674 \right) \right\}$ as the real solutions to \eqref{tc1}.

\subsection{Test Case 2}

\begin{align*}
h_1(\textbf{x}) &= -1 - x_1 \\
h_2(\textbf{x}) &= x_1 - 1 \\
h_3(\textbf{x}) &= -1 - x_2 \\
h_4(\textbf{x}) &= x_2 - 1 \\
\nabla h_1(\textbf{x}) &= \begin{bmatrix} -1 & 0 \end{bmatrix} ^T \\
\nabla h_2(\textbf{x}) &= \begin{bmatrix} 1 & 0 \end{bmatrix} ^T \\
\nabla h_3(\textbf{x}) &= \begin{bmatrix} 0 & -1 \end{bmatrix} ^T \\
\nabla h_4(\textbf{x}) &= \begin{bmatrix} 0 & 1 \end{bmatrix} ^T \\
\end{align*}

Substituting in \eqref{kkt}, we get
\begin{align*}
2 x_1 - x_2 - 2 - \lambda_1 + \lambda_2 &= 0 \\
2 x_2 - x_1 - 2 - \lambda_3 + \lambda_4 &= 0 \\
\lambda_1 (x_1 + 1) &= 0 \\
\lambda_2 (x_1 - 1) &= 0 \\
\lambda_3 (x_2 + 1) &= 0 \\
\lambda_4 (x_2 - 1) &= 0 \labeleqn{tc2}
\end{align*}

We observe that when $\textbf{x} = \begin{bmatrix} 1 & 1 \end{bmatrix} ^T$, the system \eqref{tc2} is satisifed for $\boldsymbol{\lambda} = \begin{bmatrix} 0 & 1 & 0 & 1 \end{bmatrix} ^T$.

\subsection{Test Case 3}

\begin{align*}
h_1(\textbf{x}) &= - x_1 \\
h_2(\textbf{x}) &= x_1 - 3 \\
h_3(\textbf{x}) &= - x_2 \\
h_4(\textbf{x}) &= x_2 - 3 \\
\nabla h_1(\textbf{x}) &= \begin{bmatrix} -1 & 0 \end{bmatrix} ^T \\
\nabla h_2(\textbf{x}) &= \begin{bmatrix} 1 & 0 \end{bmatrix} ^T \\
\nabla h_3(\textbf{x}) &= \begin{bmatrix} 0 & -1 \end{bmatrix} ^T \\
\nabla h_4(\textbf{x}) &= \begin{bmatrix} 0 & 1 \end{bmatrix} ^T \\
\end{align*}

Substituting in \eqref{kkt}, we get
\begin{align*}
2 x_1 - x_2 - 2 - \lambda_1 + \lambda_2 &= 0 \\
2 x_2 - x_1 - 2 - \lambda_3 + \lambda_4 &= 0 \\
\lambda_1 x_1 &= 0 \\
\lambda_2 (x_1 - 3) &= 0 \\
\lambda_3 x_2 &= 0 \\
\lambda_4 (x_2 - 3) &= 0 \labeleqn{tc3}
\end{align*}

We observe that when $\textbf{x} = \begin{bmatrix} 2 & 2 \end{bmatrix} ^T$, the system \eqref{tc3} is satisifed for $\boldsymbol{\lambda} = \begin{bmatrix} 0 & 0 & 0 & 0 \end{bmatrix} ^T$.

\subsection{Test Case 4}

\begin{align*}
h_1(\textbf{x}) &= 3 - x_1 \\
h_2(\textbf{x}) &= x_1 - 4 \\
h_3(\textbf{x}) &= 3 - x_2 \\
h_4(\textbf{x}) &= x_2 - 4 \\
\nabla h_1(\textbf{x}) &= \begin{bmatrix} -1 & 0 \end{bmatrix} ^T \\
\nabla h_2(\textbf{x}) &= \begin{bmatrix} 1 & 0 \end{bmatrix} ^T \\
\nabla h_3(\textbf{x}) &= \begin{bmatrix} 0 & -1 \end{bmatrix} ^T \\
\nabla h_4(\textbf{x}) &= \begin{bmatrix} 0 & 1 \end{bmatrix} ^T \\
\end{align*}

Substituting in \eqref{kkt}, we get
\begin{align*}
2 x_1 - x_2 - 2 - \lambda_1 + \lambda_2 &= 0 \\
2 x_2 - x_1 - 2 - \lambda_3 + \lambda_4 &= 0 \\
\lambda_1 (x_1 - 3) &= 0 \\
\lambda_2 (x_1 - 4) &= 0 \\
\lambda_3 (x_2 - 3) &= 0 \\
\lambda_4 (x_2 - 4) &= 0 \labeleqn{tc4}
\end{align*}

We observe that when $\textbf{x} = \begin{bmatrix} 3 & 3 \end{bmatrix} ^T$, the system \eqref{tc4} is satisifed for $\boldsymbol{\lambda} = \begin{bmatrix} 1 & 0 & 1 & 0 \end{bmatrix} ^T$.

\section{Matyas Function}

$$f(\textbf{x}) = 0.26(x_1^2 + x_2^2) - 0.48 x_1 x_2$$

\begin{align*}
\nabla f(\textbf{x}) &=
    \begin{bmatrix}
        0.52 x_1 - 0.48 x_2 \\
        0.52 x_2 - 0.48 x_1
    \end{bmatrix}
\end{align*}

\subsection{Test Case 5}

\begin{align*}
h_1(\textbf{x}) &= - x_1 \\
h_2(\textbf{x}) &= x_1 - 1 \\
h_3(\textbf{x}) &= - x_2 \\
h_4(\textbf{x}) &= x_2 - 1 \\
\nabla h_1(\textbf{x}) &= \begin{bmatrix} -1 & 0 \end{bmatrix} ^T \\
\nabla h_2(\textbf{x}) &= \begin{bmatrix} 1 & 0 \end{bmatrix} ^T \\
\nabla h_3(\textbf{x}) &= \begin{bmatrix} 0 & -1 \end{bmatrix} ^T \\
\nabla h_4(\textbf{x}) &= \begin{bmatrix} 0 & 1 \end{bmatrix} ^T
\end{align*}

Substituting in \eqref{kkt}, we get
\begin{align*}
0.52 x_1 - 0.48 x_2 - \lambda_1 + \lambda_2 &= 0 \\
0.52 x_2 - 0.48 x_1 - \lambda_3 + \lambda_4 &= 0 \\
\lambda_1 x_1 &= 0 \\
\lambda_2 (x_1 - 1) &= 0 \\
\lambda_3 x_2 &= 0 \\
\lambda_4 (x_2 - 1) &= 0 \labeleqn{tc5}
\end{align*}

We observe that when $\textbf{x} = \begin{bmatrix} 0 & 0 \end{bmatrix} ^T$, the system \eqref{tc5} is satisifed for $\boldsymbol{\lambda} = \begin{bmatrix} 0 & 0 & 0 & 0 \end{bmatrix} ^T$.

\subsection{Test Case 6}

\begin{align*}
h_1(\textbf{x}) &= 1 - x_1 \\
h_2(\textbf{x}) &= x_1 - 2 \\
h_3(\textbf{x}) &= 1 - x_2 \\
h_4(\textbf{x}) &= x_2 - 2 \\
\nabla h_1(\textbf{x}) &= \begin{bmatrix} -1 & 0 \end{bmatrix} ^T \\
\nabla h_2(\textbf{x}) &= \begin{bmatrix} 1 & 0 \end{bmatrix} ^T \\
\nabla h_3(\textbf{x}) &= \begin{bmatrix} 0 & -1 \end{bmatrix} ^T \\
\nabla h_4(\textbf{x}) &= \begin{bmatrix} 0 & 1 \end{bmatrix} ^T
\end{align*}

Substituting in \eqref{kkt}, we get
\begin{align*}
0.52 x_1 - 0.48 x_2 - \lambda_1 + \lambda_2 &= 0 \\
0.52 x_2 - 0.48 x_1 - \lambda_3 + \lambda_4 &= 0 \\
\lambda_1 (x_1 - 1) &= 0 \\
\lambda_2 (x_1 - 2) &= 0 \\
\lambda_3 (x_2 - 1) &= 0 \\
\lambda_4 (x_2 - 2) &= 0 \labeleqn{tc6}
\end{align*}

We observe that when $\textbf{x} = \begin{bmatrix} 1 & 1 \end{bmatrix} ^T$, the system \eqref{tc6} is satisifed for $\boldsymbol{\lambda} = \begin{bmatrix} 0.04 & 0 & 0.04 & 0 \end{bmatrix} ^T$.

\subsection{Test Case 7}

\begin{align*}
h_1(\textbf{x}) &= -1 - x_1 \\
h_2(\textbf{x}) &= x_1 + 0.5 \\
h_3(\textbf{x}) &= -0.5 - x_2 \\
h_4(\textbf{x}) &= x_2 - 0.5 \\
\nabla h_1(\textbf{x}) &= \begin{bmatrix} -1 & 0 \end{bmatrix} ^T \\
\nabla h_2(\textbf{x}) &= \begin{bmatrix} 1 & 0 \end{bmatrix} ^T \\
\nabla h_3(\textbf{x}) &= \begin{bmatrix} 0 & -1 \end{bmatrix} ^T \\
\nabla h_4(\textbf{x}) &= \begin{bmatrix} 0 & 1 \end{bmatrix} ^T
\end{align*}

Substituting in \eqref{kkt}, we get
\begin{align*}
0.52 x_1 - 0.48 x_2 - \lambda_1 + \lambda_2 &= 0 \\
0.52 x_2 - 0.48 x_1 - \lambda_3 + \lambda_4 &= 0 \\
\lambda_1 (x_1 + 1) &= 0 \\
\lambda_2 (x_1 + 0.5) &= 0 \\
\lambda_3 (x_2 + 0.5) &= 0 \\
\lambda_4 (x_2 - 0.5) &= 0 \labeleqn{tc7}
\end{align*}

We observe that when $\textbf{x} = \begin{bmatrix} -0.5 & -0.4615 \end{bmatrix} ^T$, the system \eqref{tc7} is satisifed for $\boldsymbol{\lambda} = \begin{bmatrix} 0 & 0.03848 & 0 & 0 \end{bmatrix} ^T$.

\end{document}
