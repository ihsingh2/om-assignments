\documentclass[a4paper]{article}
\usepackage[margin=1.25in]{geometry}
\usepackage{bookmark}

\usepackage{amsmath}
\usepackage{amssymb}
\allowdisplaybreaks
\newcommand{\numberthis}{\addtocounter{equation}{1}\tag{\theequation}}
\newcommand{\labeleqn}[1]{\numberthis \label{#1}}

\usepackage{graphicx}
\usepackage{float}

\renewcommand{\baselinestretch}{1.15}
\setlength{\parindent}{0pt}

\usepackage{multirow}
\usepackage{tabularx}
\newcolumntype{L}{>{\centering\arraybackslash}X}

\usepackage[skins]{tcolorbox}
\newtcolorbox{plainbox}[2][]{%
  enhanced,colback=white,colframe=black,coltitle=black,
  sharp corners,boxrule=0.4pt,
  fonttitle=\itshape,
  attach boxed title to top center={yshift=-0.4\baselineskip-0.4pt},
  boxed title style={tile,size=minimal,left=0.5mm,right=0.5mm,
    colback=white,before upper=\strut},
  title=#2,#1
}

\title{CS1.404: Assignment 3}
\author{Himanshu Singh}
\date{\today}

\begin{document}

\maketitle

\begin{plainbox}{KKT Condition}
For the convex optimization problem,
\begin{align*}
\min_{\textbf{x} \in \mathbb{R}^n} f(\textbf{x}) \\
\text{s.t. } h_j(\textbf{x}) \leq 0 && \forall j \in \{1, 2, \ldots, l\}
\intertext{$\textbf{x}^*$ is an optimal solution, if there exists multipliers $\{\lambda_j\}_{j = 1}^l$, such that the following equations hold.}
\nabla f(\textbf{x}^*) + \sum_{j = 1}^l \lambda_j \nabla h_j (\textbf{x}^*) = 0 \\
\lambda_j h_j (\textbf{x}^*) = 0 && \forall j \in \{1, 2, \ldots, l\} \\
\lambda_j \geq 0 && \forall j \in \{1, 2, \ldots, l\} \labeleqn{kkt}
\end{align*}
\end{plainbox}

\section{Trid Function}

$$f(\textbf{x}) = \sum_{i=1}^d (x_i - 1)^2 - \sum_{i=2}^d x_{i-1} x_i$$

\begin{align*}
\nabla f(\textbf{x}) &=
    \begin{bmatrix}
        2 x_1 - x_2 - 2 \\
        2 x_2 - x_1 - x_3 - 2 \\
        2 x_3 - x_2 - x_4 - 2 \\
        \vdots \\
        2 x_{d-1} - x_{d-2} - x_d - 2 \\
        2 x_d - x_{d-1} - 2 \\
    \end{bmatrix}
\end{align*}

\subsection{Test Case 0}

\begin{align*}
h_1(\textbf{x}) &= x_1^2 - 2 x_2 \\
\nabla h_1(\textbf{x}) &= \begin{bmatrix} 2 x_1 & -2 \end{bmatrix} ^T \\
\end{align*}

Substituting in \eqref{kkt}, we get
\begin{align*}
2 x_1 - x_2 - 2 + 2 \lambda_1 x_1 &= 0 \\
2 x_2 - x_1 - 2 - 2 \lambda_1 &= 0 \\
\lambda_1 (x_1^2 - 2 x_2) &= 0
\end{align*}

\subsection{Test Case 1}

\begin{align*}
h_1(\textbf{x}) &= x_1^2 - x_2^2 + 1 \\
\nabla h_1(\textbf{x}) &= \begin{bmatrix} 2 x_1 & -2 x_2 \end{bmatrix} ^T \\
\end{align*}

Substituting in \eqref{kkt}, we get
\begin{align*}
2 x_1 - x_2 - 2 + 2 \lambda_1 x_1 &= 0 \\
2 x_2 - x_1 - 2 - 2 \lambda_1 x_2 &= 0 \\
\lambda_1 (x_1^2 - x_2^2 + 1) &= 0
\end{align*}

\subsection{Test Case 2}

\begin{align*}
h_1(\textbf{x}) &= -1 - x_1 \\
h_2(\textbf{x}) &= x_1 - 1 \\
h_3(\textbf{x}) &= -1 - x_2 \\
h_4(\textbf{x}) &= x_2 - 1 \\
\nabla h_1(\textbf{x}) &= \begin{bmatrix} -1 & 0 \end{bmatrix} ^T \\
\nabla h_2(\textbf{x}) &= \begin{bmatrix} 1 & 0 \end{bmatrix} ^T \\
\nabla h_3(\textbf{x}) &= \begin{bmatrix} 0 & -1 \end{bmatrix} ^T \\
\nabla h_4(\textbf{x}) &= \begin{bmatrix} 0 & 1 \end{bmatrix} ^T \\
\end{align*}

Substituting in \eqref{kkt}, we get
\begin{align*}
2 x_1 - x_2 - 2 - \lambda_1 + \lambda_2 &= 0 \\
2 x_2 - x_1 - 2 - \lambda_3 + \lambda_4 &= 0 \\
\lambda_1 (x_1 + 1) &= 0 \\
\lambda_2 (x_1 - 1) &= 0 \\
\lambda_3 (x_2 + 1) &= 0 \\
\lambda_4 (x_2 - 1) &= 0
\end{align*}

\subsection{Test Case 3}

\begin{align*}
h_1(\textbf{x}) &= - x_1 \\
h_2(\textbf{x}) &= x_1 - 3 \\
h_3(\textbf{x}) &= - x_2 \\
h_4(\textbf{x}) &= x_2 - 3 \\
\nabla h_1(\textbf{x}) &= \begin{bmatrix} -1 & 0 \end{bmatrix} ^T \\
\nabla h_2(\textbf{x}) &= \begin{bmatrix} 1 & 0 \end{bmatrix} ^T \\
\nabla h_3(\textbf{x}) &= \begin{bmatrix} 0 & -1 \end{bmatrix} ^T \\
\nabla h_4(\textbf{x}) &= \begin{bmatrix} 0 & 1 \end{bmatrix} ^T \\
\end{align*}

Substituting in \eqref{kkt}, we get
\begin{align*}
2 x_1 - x_2 - 2 - \lambda_1 + \lambda_2 &= 0 \\
2 x_2 - x_1 - 2 - \lambda_3 + \lambda_4 &= 0 \\
\lambda_1 x_1 &= 0 \\
\lambda_2 (x_1 - 3) &= 0 \\
\lambda_3 x_2 &= 0 \\
\lambda_4 (x_2 - 3) &= 0
\end{align*}

\subsection{Test Case 4}

\begin{align*}
h_1(\textbf{x}) &= 3 - x_1 \\
h_2(\textbf{x}) &= x_1 - 4 \\
h_3(\textbf{x}) &= 3 - x_2 \\
h_4(\textbf{x}) &= x_2 - 4 \\
\nabla h_1(\textbf{x}) &= \begin{bmatrix} -1 & 0 \end{bmatrix} ^T \\
\nabla h_2(\textbf{x}) &= \begin{bmatrix} 1 & 0 \end{bmatrix} ^T \\
\nabla h_3(\textbf{x}) &= \begin{bmatrix} 0 & -1 \end{bmatrix} ^T \\
\nabla h_4(\textbf{x}) &= \begin{bmatrix} 0 & 1 \end{bmatrix} ^T \\
\end{align*}

Substituting in \eqref{kkt}, we get
\begin{align*}
2 x_1 - x_2 - 2 - \lambda_1 + \lambda_2 &= 0 \\
2 x_2 - x_1 - 2 - \lambda_3 + \lambda_4 &= 0 \\
\lambda_1 (x_1 - 3) &= 0 \\
\lambda_2 (x_1 - 4) &= 0 \\
\lambda_3 (x_2 - 3) &= 0 \\
\lambda_4 (x_2 - 4) &= 0
\end{align*}

\section{Matyas Function}

$$f(\textbf{x}) = 0.26(x_1^2 + x_2^2) - 0.48 x_1 x_2$$

\begin{align*}
\nabla f(\textbf{x}) &=
    \begin{bmatrix}
        0.52 x_1 - 0.48 x_2 \\
        0.52 x_2 - 0.48 x_1
    \end{bmatrix}
\end{align*}

\subsection{Test Case 5}

\begin{align*}
h_1(\textbf{x}) &= - x_1 \\
h_2(\textbf{x}) &= x_1 - 1 \\
h_3(\textbf{x}) &= - x_2 \\
h_4(\textbf{x}) &= x_2 - 1 \\
\nabla h_1(\textbf{x}) &= \begin{bmatrix} -1 & 0 \end{bmatrix} ^T \\
\nabla h_2(\textbf{x}) &= \begin{bmatrix} 1 & 0 \end{bmatrix} ^T \\
\nabla h_3(\textbf{x}) &= \begin{bmatrix} 0 & -1 \end{bmatrix} ^T \\
\nabla h_4(\textbf{x}) &= \begin{bmatrix} 0 & 1 \end{bmatrix} ^T
\end{align*}

Substituting in \eqref{kkt}, we get
\begin{align*}
0.52 x_1 - 0.48 x_2 - \lambda_1 + \lambda_2 &= 0 \\
0.52 x_2 - 0.48 x_1 - \lambda_3 + \lambda_4 &= 0 \\
\lambda_1 x_1 &= 0 \\
\lambda_2 (x_1 - 1) &= 0 \\
\lambda_3 x_2 &= 0 \\
\lambda_4 (x_2 - 1) &= 0
\end{align*}

\subsection{Test Case 6}

\begin{align*}
h_1(\textbf{x}) &= 1 - x_1 \\
h_2(\textbf{x}) &= x_1 - 2 \\
h_3(\textbf{x}) &= 1 - x_2 \\
h_4(\textbf{x}) &= x_2 - 2 \\
\nabla h_1(\textbf{x}) &= \begin{bmatrix} -1 & 0 \end{bmatrix} ^T \\
\nabla h_2(\textbf{x}) &= \begin{bmatrix} 1 & 0 \end{bmatrix} ^T \\
\nabla h_3(\textbf{x}) &= \begin{bmatrix} 0 & -1 \end{bmatrix} ^T \\
\nabla h_4(\textbf{x}) &= \begin{bmatrix} 0 & 1 \end{bmatrix} ^T
\end{align*}

Substituting in \eqref{kkt}, we get
\begin{align*}
0.52 x_1 - 0.48 x_2 - \lambda_1 + \lambda_2 &= 0 \\
0.52 x_2 - 0.48 x_1 - \lambda_3 + \lambda_4 &= 0 \\
\lambda_1 (x_1 - 1) &= 0 \\
\lambda_2 (x_1 - 2) &= 0 \\
\lambda_3 (x_2 - 1) &= 0 \\
\lambda_4 (x_2 - 2) &= 0
\end{align*}

\subsection{Test Case 7}

\begin{align*}
h_1(\textbf{x}) &= -1 - x_1 \\
h_2(\textbf{x}) &= x_1 + 0.5 \\
h_3(\textbf{x}) &= -0.5 - x_2 \\
h_4(\textbf{x}) &= x_2 - 0.5 \\
\nabla h_1(\textbf{x}) &= \begin{bmatrix} -1 & 0 \end{bmatrix} ^T \\
\nabla h_2(\textbf{x}) &= \begin{bmatrix} 1 & 0 \end{bmatrix} ^T \\
\nabla h_3(\textbf{x}) &= \begin{bmatrix} 0 & -1 \end{bmatrix} ^T \\
\nabla h_4(\textbf{x}) &= \begin{bmatrix} 0 & 1 \end{bmatrix} ^T
\end{align*}

Substituting in \eqref{kkt}, we get
\begin{align*}
0.52 x_1 - 0.48 x_2 - \lambda_1 + \lambda_2 &= 0 \\
0.52 x_2 - 0.48 x_1 - \lambda_3 + \lambda_4 &= 0 \\
\lambda_1 (x_1 + 1) &= 0 \\
\lambda_2 (x_1 + 0.5) &= 0 \\
\lambda_3 (x_2 + 0.5) &= 0 \\
\lambda_4 (x_2 - 0.5) &= 0
\end{align*}

\end{document}
